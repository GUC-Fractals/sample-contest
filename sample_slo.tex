
\documentclass[a4paper]{article}
\usepackage[english]{babel}
\usepackage{graphicx}
\usepackage{multicol}
\usepackage{amsmath}
\usepackage{hyperref}
\usepackage{amsthm}
\usepackage{geometry}
\geometry{a4paper}
\usepackage{fancyhdr}
\usepackage{xcolor}
\usepackage{amssymb}
\usepackage{multicol}
\theoremstyle{definition}
\newtheorem{exmp}{Example}[section]
\usepackage{tcolorbox}

\begin{document}
\author{Ibrahim Abou Elenein}
\title{\textbf{Sample Contest}}
\maketitle
\noindent
\begin{enumerate}


\item Define The operation $a@b$ to be $3 + ab + a + 2b$. There exists a number
$x$ such that $x@b =1$ for all $b$. Find $x$.
\begin{tcolorbox}[width=\linewidth, sharp corners=all, colback=white!95!black]

$\textit{ Sol. }$

\[
    3 + xb + x + 2b = 1
\]
\[
    xb + x + 2b + 2 = 0 \Rightarrow (x+2)(b+1) = 0
\]
We see that if $x=-2$, then this expression is always true. Indeed
plugging $-2$ for $x$ yeilds $3-2b-2+2b=1$ so The answer is $-2$

\end{tcolorbox}

\item Let $y = x^2 + bx +c$ be a quadratic function. it has only one real root. if $b$
is postive, find $\dfrac{b+2}{\sqrt{c} + 1}$.

\begin{tcolorbox}[width=\linewidth, sharp corners=all, colback=white!95!black]

$\textit{ sol. }$

since it has only one real root, we know that the discriminant $b^2-4ac$ is $0$
since $a=1$, we have $b^2 = 4c$ thus $\dfrac{b+2}{\sqrt{c} + 1} = \dfrac{2\sqrt{c} + 2}{\sqrt{c} + 1} = 2$
\end{tcolorbox}


\item A circle of nonzero radius r has a circumference numerically equal to
$\dfrac{1}{3}$ of its area. What is its area?

\begin{tcolorbox}[width=\linewidth, sharp corners=all, colback=white!95!black]
    $  \frac{1}{3} $Area =  $\frac{1}{3} \pi r^2$ = $2\pi r$
    \[
       \Rightarrow r = 6 \Rightarrow \text{Area} = \pi r^2 = \pi 6^2 = 36\pi
    \]
\end{tcolorbox}

\item Let set $\mathcal{A}$ be a 90-element subset of $\{1,2,3,\ldots,100\},$ and let $S$ be the sum of the elements of $\mathcal{A}.$ Find the number of possible values of $S.$

\begin{tcolorbox}[width=\linewidth, sharp corners=all, colback=white!95!black]
    The smallest $S$ is $1+2+ \ldots +90 = 91 \cdot 45 = 4095$. The largest $S$
    is $11+12+ \ldots +100=111\cdot 45=4995$. All numbers between $4095$ and
    $4995$ are possible values of S, so the number of possible values of S is
    $4995-4095+1=901$.

\end{tcolorbox}




\item A \textit{gorgeous} sequence is a sequence of 1’s and 0’s such that there are no consecutive 1’s. For
instance, the set of all gorgeous sequences of length 3 is {[1, 0, 0], [1, 0, 1], [0, 1, 0], [0, 0, 1], [0, 0, 0]}.
Determine the number of gorgeous sequences of length 7.

\begin{tcolorbox}[width=\linewidth, sharp corners=all, colback=white!95!black]
    Let $S_n$ be the number of $gorgeous$ sequences of length $n$.
    Looking at aribtary sequence of length $n$, we see that
    it either strats with $1$ or $0$. If starts with $0$,
    Then we can simply take all sequences of length $n-1$ and
    append it to the $0$.
    If it starts with $1$, Then the next value must be. After
    that we can take all sequences of length $n-2$ and append it
    to the $[0, 1]$. So $S_n = S_{n-1} + S_{n-2}$
    The first two terms are $S_0 = 1$ and $S_1 = 2$
    so $S_n$ is just fibonacci numbers. so recuresivly we have
    $S_7 = 34$
\end{tcolorbox}
 \item A \(8 \times 8\) chessboard with the northeast and southwest corner unit squares removed is given. Is it
          possible to partition such a a board into thirty-one unit dominoes(where a domino is a \(1 \times 2\) rectangle)?
          Show your work.

\begin{tcolorbox}[width=\linewidth, sharp corners=all, colback=white!95!black]
    For such board, we can color the unit squares alternatively.
    in black and white.
    Each domino will cover two adjacent squares one with color
    balck and one with color white. So if 31 dominoes can over the board
    there should be 31 squares with black color and 31 squares with white color.
    However, in the board we have  32 squares of color balck
    and 30 of white, So the task is impossible.
\end{tcolorbox}
\item The function $f$ satifies
\[
    f(x) + f(2x+y) + 5xy = f(3x - y) + 2x^2 + 1
\]
for all real numbers $x, y$. Determine the vlaue of $f(10)$

\begin{tcolorbox}[width=\linewidth, sharp corners=all, colback=white!95!black]
    Setting $x = 10$ and $y= 5$ gives
    \[
        f(10) + f(25)+ 250  =f(25) + 200 +1
    \]
    from which gives $f(10) = -49$

\end{tcolorbox}
\item Let
\[
    a = \underbrace{19191919191\dots1919}_\text{19 is repeted 3838 times}
\]
What is the reminder of $a$ when divided by $13$?
\begin{tcolorbox}[width=\linewidth, sharp corners=all, colback=white!95!black]
    Not that $13 | 191919$. Thus $13 | \underbrace{1919191\dots1919100}_{19
    \text{ is repeated 3837 times} }$ since $3| 3837$.
    Howeveer $a = \underbrace{19191919 \dots 191900 }_{\text{19 is repeated 3837}}+ 19$ so the reminder
    whwn $a$ is divided by $13$
    is the same as the reminder when $19$ is divided by $13$ which is 6.
\end{tcolorbox}
\end{enumerate}
\end{document}
